% Template for PLoS
% Version 3.1 February 2015
%
% To compile to pdf, run:
% latex plos.template
% bibtex plos.template
% latex plos.template
% latex plos.template
% dvipdf plos.template
%
% % % % % % % % % % % % % % % % % % % % % %
%
% -- IMPORTANT NOTE
%
% This template contains comments intended
% to minimize problems and delays during our production
% process. Please follow the template instructions
% whenever possible.
%
% % % % % % % % % % % % % % % % % % % % % % %
%
% Once your paper is accepted for publication,
% PLEASE REMOVE ALL TRACKED CHANGES in this file and leave only
% the final text of your manuscript.
%
% There are no restrictions on package use within the LaTeX files except that
% no packages listed in the template may be deleted.
%
% Please do not include colors or graphics in the text.
%
% Please do not create a heading level below \subsection. For 3rd level headings, use \paragraph{}.
%
% % % % % % % % % % % % % % % % % % % % % % %
%
% -- FIGURES AND TABLES
%
% Please include tables/figure captions directly after the paragraph where they are first cited in the text.
%
% DO NOT INCLUDE GRAPHICS IN YOUR MANUSCRIPT
% - Figures should be uploaded separately from your manuscript file.
% - Figures generated using LaTeX should be extracted and removed from the PDF before submission.
% - Figures containing multiple panels/subfigures must be combined into one image file before submission.
% For figure citations, please use "Fig." instead of "Figure".
% See http://www.plosone.org/static/figureGuidelines for PLOS figure guidelines.
%
% Tables should be cell-based and may not contain:
% - tabs/spacing/line breaks within cells to alter layout or alignment
% - vertically-merged cells (no tabular environments within tabular environments, do not use \multirow)
% - colors, shading, or graphic objects
% See http://www.plosone.org/static/figureGuidelines#tables for table guidelines.
%
% For tables that exceed the width of the text column, use the adjustwidth environment as illustrated in the example table in text below.
%
% % % % % % % % % % % % % % % % % % % % % % % %
%
% -- EQUATIONS, MATH SYMBOLS, SUBSCRIPTS, AND SUPERSCRIPTS
%
% IMPORTANT
% Below are a few tips to help format your equations and other special characters according to our specifications. For more tips to help reduce the possibility of formatting errors during conversion, please see our LaTeX guidelines at http://www.plosone.org/static/latexGuidelines
%
% Please be sure to include all portions of an equation in the math environment.
%
% Do not include text that is not math in the math environment. For example, CO2 will be CO\textsubscript{2}.
%
% Please add line breaks to long display equations when possible in order to fit size of the column.
%
% For inline equations, please do not include punctuation (commas, etc) within the math environment unless this is part of the equation.
%
% % % % % % % % % % % % % % % % % % % % % % % %
%
% Please contact latex@plos.org with any questions.
%
% % % % % % % % % % % % % % % % % % % % % % % %

\documentclass[10pt,letterpaper]{article}
\usepackage[top=0.85in,left=2.75in,footskip=0.75in]{geometry}

% Use adjustwidth environment to exceed column width (see example table in text)
\usepackage{changepage}

% Use Unicode characters when possible
\usepackage[utf8]{inputenc}

% textcomp package and marvosym package for additional characters
\usepackage{textcomp,marvosym}

% fixltx2e package for \textsubscript
\usepackage{fixltx2e}

% amsmath and amssymb packages, useful for mathematical formulas and symbols
\usepackage{amsmath,amssymb}

% cite package, to clean up citations in the main text. Do not remove.
\usepackage{cite}

% Use nameref to cite supporting information files (see Supporting Information section for more info)
\usepackage{nameref,hyperref}

% line numbers
\usepackage[right]{lineno}

% ligatures disabled
\usepackage{microtype}
%\DisableLigatures[f]{encoding = *, family = * }

% rotating package for sideways tables
\usepackage{rotating}

\usepackage{tikz}
\usepackage{subfig}
\usepackage{algorithm}
\usepackage{algpseudocode}
\usepackage{calc}
\usepackage{siunitx}
%\usepackage{graphics}
\graphicspath{{figures/upload/}}  % added by JMP

% Remove comment for double spacing
%\usepackage{setspace}
%\doublespacing

% Text layout
\raggedright
\setlength{\parindent}{0.5cm}
\textwidth 5.25in
\textheight 8.75in

% Bold the 'Figure #' in the caption and separate it from the title/caption with a period
% Captions will be left justified
\usepackage[aboveskip=1pt,labelfont=bf,labelsep=period,justification=raggedright,singlelinecheck=off]{caption}

% Use the PLoS provided BiBTeX style
\bibliographystyle{plos2015}

% Remove brackets from numbering in List of References
\makeatletter
\renewcommand{\@biblabel}[1]{\quad#1.}
\makeatother

% Leave date blank
\date{}

% Header and Footer with logo
\usepackage{lastpage,fancyhdr,graphicx}
\usepackage{epstopdf}
\pagestyle{myheadings}
\pagestyle{fancy}
\fancyhf{}
%\lhead{\includegraphics[width=2.0in]{PLOS-submission.eps}}
\fancyhead[R]{\fontsize{12}{14} \selectfont Neuron's Eye View: Inferring Features of Complex Stimuli from Neural Responses}
\rfoot{\thepage/\pageref{LastPage}}
\renewcommand{\footrule}{\hrule height 2pt \vspace{2mm}}
\fancyheadoffset[L]{2.25in}
\fancyfootoffset[L]{2.25in}
\lfoot{\sf PLOS}

%% Include all macros below

\newcommand{\lorem}{{\bf LOREM}}
\newcommand{\ipsum}{{\bf IPSUM}}

% Generelt
\newcommand{\ud}{\mathrm{d}} % d
\newcommand{\R}{\mathbb{R}} % Real numbers
\newcommand{\T}{\mathscr{T}}

% Operatorer og funktionaler
\newcommand{\I}{\mathbb{I}} % I

\newcommand{\CE}[2]{\mathrm{E}\left[\,#1\,|\,#2\,\right]} % Conditional expectation

\newcommand{\by}{\boldsymbol{y}}
\newcommand{\bc}{\boldsymbol{c}}
\newcommand{\bd}{\boldsymbol{d}}
\newcommand{\bP}{\boldsymbol{\Phi}}
\newcommand{\bZ}{Z}%{\boldsymbol{Z}}
\newcommand{\bV}{V}%{\boldsymbol{V}}
\newcommand{\br}{\boldsymbol{r}}
\newcommand{\bt}{\boldsymbol{t}}
\newcommand{\btheta}{\boldsymbol{\vartheta}}
\newcommand{\bht}{\hat{\boldsymbol{\vartheta}}}
\newcommand{\bz}{\boldsymbol{z}}
\newcommand{\bx}{{\boldsymbol{x}}}
\newcommand{\bv}{v}
\newcommand{\bw}{\boldsymbol{w}}
\newcommand{\vv}{\boldsymbol{v}}
\newcommand{\bepsilon}{\boldsymbol{\varepsilon}}

%% END MACROS SECTION

\setcounter{secnumdepth}{3}
\begin{document}
\vspace*{0.35in}

% Title must be 250 characters or less.
% Please capitalize all terms in the title except conjunctions, prepositions, and articles.
\begin{flushleft}
{\Large
\textbf{S3 Text. Effects of bin size and dynamics\\}
% \bigskip
% \textbf{Effects of Bin Size and Dynamics} % Please use "title case" (capitalize all terms in the title except conjunctions, prepositions, and articles).
}
% \newline
% % Insert author names, affiliations and corresponding author email (do not include titles, positions, or degrees).
% Xin (Cindy) Chen\textsuperscript{1,3},
% Jeffrey M. Beck\textsuperscript{2,3},
% John M. Pearson\textsuperscript{1,3*},
% \\
% \bigskip
% \textbf{1} Duke Institute for Brain Sciences, Duke University, Durham, North Carolina, USA
% \\
% \textbf{2} Department of Neurobiology, Duke University Medical Center, Durham, North Carolina, USA
% \\
% \textbf{3} Center for Cognitive Neuroscience, Duke University, Durham, North Carolina, USA
% \\
% \bigskip

% Insert additional author notes using the symbols described below. Insert symbol callouts after author names as necessary.
%
% Remove or comment out the author notes below if they aren't used.
%
% Primary Equal Contribution Note
%\Yinyang These authors contributed equally to this work.

% Deceased author note
%\dag Deceased

% Group/Consortium Author Note
%\textpilcrow Membership list can be found in the Acknowledgments section.

% Use the asterisk to denote corresponding authorship and provide email address in note below.
% * john.pearson@duke.edu

\end{flushleft}
% Please keep the abstract below 300 words

% Place figure captions after the first paragraph in which they are cited.

% Code to reproduce all experiments presented here is available at
 % \url{https://github.com/pearsonlab/spiketopics}.

\section*{Robustness of inference to time step size}
Our model assumes a fixed time step $\Delta t$ between count observations, which is also the time step for the Hidden Markov Model. However, if the time step used for inference (and binning spike data) is too large, the model may miss important temporal features of the data. Here, we performed two experiments to test the robustness of our inferred features to these possibilities. In both cases, we simulated data from our model ($U=20$ units, $T=1$s stimulus length, $M=200$ trials, $\Delta t = 5$ms, $K_{data} = 3$ features, $K = 5$ inferred features), similar to our synthetic data experiment. Just as in that experiment, our goal was to compare our recovered latent features to those that generated the data. However, in our inference, we systematically varied $\Delta t = (5, 10, 20, 50, 100)$ ms. We used the same relative error tolerance for training as in our synthetic data experiment, with 10 random parameter restarts, from which we chose the solution with highest ELBO. We did not include overdispersion effects in this experiment, though similar results may be obtained in that condition by including a larger number of trials.

Results are displayed in Figure \ref{fig:no_trans}. At the top, we display the binary features used to generate the data. Below this, we show recovered binary features for each bin size, weighted by the mean absolute effect size (as a percent change from baseline) across the units in our simulation. This weighting serves to make inferred features with negligible effect on firing less visually salient, since a feature may be inferred with a low accompanying firing rate, effectively rendering it unused. Clearly, even though some discovered features are inverted, the model faithfully recovers the latent dynamics for small bin sizes (up to 20ms), gracefully degrading for larger bin sizes. This effect is likewise obvious in the comparison of actual, empirical, and recovered firing rates in Figure \ref{fig:no_trans_fr}, where deviations become more apparent as the time step increases and the model's recovered firing becomes coarser.


\begin{figure}[!ht]
    \includegraphics[width=.9\linewidth]{bin_expt_no_trans}
	\caption{\bf Recovery of latent features with different time step sizes.}
    Comparison of recovered latent features from synthetic data generated with a 5ms time step resolution. Top panel shows actual binary features, while bottom panels show recovered features (weighted by mean absolute effect size) for a range of increasing time step sizes.
	\label{fig:no_trans}
\end{figure}

\begin{figure}[!ht]
    \includegraphics[width=.75\linewidth]{fr_recover}
	\caption{\bf Recovery of firing rates with different time step sizes.}
    Comparison of actual firing rates, trial-averaged empirical firing rates, and model-recovered firing rates for a variety of time step sizes.
	\label{fig:no_trans_fr}
\end{figure}

\section*{Robustness of inference to firing rate dynamics}
As a second check of model robustness, we repeated the bin size experiment of the previous but added a transient increase in firing rate (e.g. stimulus onset response) to the beginning of each trial. This transient was encoded as an additional multiplicative effect of $1 + 0.1\Gamma(5, 80)$, where $\Gamma(a, b)$ is the pdf of the Gamma distribution (in the shape, rate parameterization) and the data generating $\Delta t$ was again 5ms.

Figure \ref{fig:with_trans} shows the results in the same format as Figure \ref{fig:no_trans}. Here, the model correctly recovers the underlying features for small time bin sizes, accounting for the transient by utilizing extra features (3 and 4) in the model. That is, in this simplified scenario, the model's response to additional temporal dynamics is simply to infer additional features. Nevertheless, as in the previous section, with larger time steps, features begin to blend, and the transient dynamics is poorly captured. This is also visible in the corresponding firing rate reconstruction (Figure \ref{fig:with_trans_fr})

\begin{figure}[!ht]
    \includegraphics[width=.9\linewidth]{bin_expt_with_trans}
	\caption{\bf Recovery of latent features with transient dynamics.}
    Comparison of recovered latent features from synthetic data generated with a 5ms time step resolution and a transient increase in firing at the start of each trial. Top panel shows actual binary features, while bottom panels show recovered features (weighted by mean absolute effect size) for a range of increasing time step sizes.
	\label{fig:with_trans}
\end{figure}

\begin{figure}[!ht]
    \includegraphics[width=.75\linewidth]{fr_recover_trans}
	\caption{\bf Recovery of firing rates with transient dynamics.}
    Comparison of actual firing rates, trial-averaged empirical firing rates, and model-recovered firing rates for a variety of time step sizes. Cf. Figure \ref{fig:no_trans_fr}, which shares the same latent dynamics but lacks the transient.
	\label{fig:with_trans_fr}
\end{figure}

\end{document}
