\documentclass{article} % For LaTeX2e
\usepackage{nips15submit_e,times}
\usepackage{hyperref}
\usepackage{url}
\usepackage{amssymb,amsmath}
\usepackage[pdftex]{graphicx}
%\documentstyle[nips14submit_09,times,art10]{article} % For LaTeX 2.09


\title{Variational Bayesian Inference for Neural Feature Tagging of Unstructured Stimuli\\Supplemental Material}


\author{
John M.~Pearson\\
Duke Institute for Brain Sciences \\
Duke University\\
Durham, NC 27708 \\
\texttt{john.pearson@duke.edu} \\
\And
Jeffrey M.~Beck \\
Department of Neurobiology \\
Duke University Medical Center \\
Durham, NC 27708 \\
\texttt{jeff.beck@duke.edu} \\
}

\newcommand{\fix}{\marginpar{FIX}}
\newcommand{\new}{\marginpar{NEW}}

% \nipsfinalcopy % Uncomment for camera-ready version

\begin{document}

\maketitle

\section{Evidence Lower Bound (ELBO)}
Here we derive the evidence lower bound (ELBO) used as a variational objective by our inference algorithm. That is, we want to calculate
\begin{equation}
    \mathcal{L} \equiv \mathbb{E}_q \left[\log \frac{p(\Theta|N)}{q(\Theta)} \right] = \mathbb{E}_q \left[\log p(\Theta|N) \right] + \mathcal{H}[q(\Theta)]
\end{equation}
From \cite{beal2003variational}, this can be written
\begin{multline}
    \mathcal{L} = \mathbb{E}_{q(\pi)} \left[\log \frac{p(\pi)}{q(\pi)} \right] 
    + \mathbb{E}_{q(A)} \left[\log \frac{p(A)}{q(A)} \right] 
    + \mathbb{E}_{q}\left[ \log \frac{p(N, z|\lambda, A, \pi)}{q(z)}\right] \\ 
    + \mathbb{E}_{q(\theta)} \left[\log \frac{p(\theta)}{q(\theta)} \right] 
    + \mathbb{E}_{q(\lambda)} \left[\log \frac{p(\lambda)}{q(\lambda)} \right] 
    + \mathbb{E}_{q(\gamma)} \left[\log \frac{p(\gamma)}{q(\gamma)} \right] 
\end{multline}
For the first two terms, updates are standard and covered in \cite{beal2003variational}. The rest we do piece-by-piece below:

\subsection{Log evidence}
We would like to calculate $\mathbb{E}_{q}\left[ \log \frac{p(N, z|x, \Theta)}{q(z)}\right]$. To do this, we make use of expectations calculated via the posteriors returned from the forward-backward algorithm
\begin{align}
    \xi_{t} &\equiv p(z_{t}|N, \theta) &
    \Xi_{t, ij} &\equiv p(z_{t+1} = j, z_{t} = i|N, \theta) &
    \log Z_{t} &= \log p(N_{t+1}|N_{t}, \Theta) 
\end{align}
Here, we have suppressed the latent feature index $k$ and abuse notation by writing the observation index as $t$, but in the case of multiple observations at a given time, we pool across units and presentations: $N_t \equiv \sum_{m; t(m) = t} N_m$. From this, we can write
\begin{multline}
     \mathbb{E}_{q}\left[ \log p(N, z|x, \Theta) \right] = 
     \sum_{mkr} \left[N_m \left( 
        \overline{\log \theta_m} + \overline{\log \lambda_{0u(m)}} +
        \xi_{t(m)k}\overline{\log \lambda_{zuk}} + 
        x_{t(m)r} \overline{\log \lambda_{xu(m)r}}
     \right) \right] \\ 
     - \sum_m \overline{\theta_m} \mathbb{E}_q\left[\Lambda_{t(m)u(m)} \right]      + \sum_{tk} \left[ \text{tr}\left(\Xi_{tk} \overline{\log A_k^T} \right)
     + \xi_{0k}^T \overline{\log \pi_k}
     \right]
    + \text{constant}
 \end{multline} 
In what follows, we will drop the irrelevant constant. For $\overline{\log y}$, where $y \in \lbrace \theta, \lambda_0, \lambda_z, \lambda_x \rbrace$, the assumption $q(y) = \text{Ga}(\alpha, \beta)$ gives
\begin{equation}
    \overline{\log y} = \psi(\alpha) - \log \beta
\end{equation}
with $\psi(x)$ the digamma function. Likewise, the expectation $\overline{\theta}$ is straightforward. For the expectation of the rate, we have
\begin{equation}
    \mathbb{E}_q\left[
        \lambda_{0u} \prod_k \lambda_{zuk}^{z_{tk}} \prod_r \lambda_{xur}^{x_{tr}}
    \right] = \frac{\alpha_{0u}}{\beta_{0u}}
    \prod_k \left(1 - \xi_{tk} + \xi_{tk} \frac{\alpha_{zuk}}{\beta_{zuk}} \right)
    \prod_r \frac{1}{\beta_{xur}^{x_{tr}}} \frac{\Gamma(\alpha_{xur} + x_{tr})}{\Gamma(\alpha_{xur})}
\end{equation}
However, for $\alpha \gg 1$, we can write $\Gamma(\alpha + x)/\Gamma(\alpha) \approx \alpha^\gamma$, so that we can write 
\begin{equation}
    \mathbb{E}_q[\Lambda_{tu}] = H_{0u} F_{tu} G_{tu}
\end{equation}
with $G_{tu} \approx \prod_r (\alpha_{xur}/\beta_{xur})^{x_{tr}}$. In addition, it will later be useful to have the expectation over \emph{all except} a particular feature $k$ or $r$, for which we define
\begin{align}
    F_{tuk} &\equiv \prod_{k'\neq k} \left(1 - \xi_{tk'} + \xi_{tk'} \frac{\alpha_{zuk'}}{\beta_{zuk'}} \right) \\
    G_{tur} &\equiv \prod_{r' \neq r} \left(\frac{\alpha_{xur'}}{\beta_{xur'}} \right)^{x_{tr'}}
\end{align}
Finally, we want the entropy of the variational posterior over $z$, $\mathbb{E}_q[-\log q(z)]$. We can write this in the form
\begin{equation}
    -\sum_{tk} \left[
        \xi_{tk}^T\eta_{tk} + \text{tr}\left(\Xi_{tk} \tilde{A}_k^T \right)
        + \xi_{0k}^T\tilde{\pi}_k
        - \log Z_{tk}
    \right]
\end{equation}
with $(\eta, \tilde{A}, \tilde{\pi})$ the parameters of the variational posterior corresponding to the emission, transition, and initial state probabilities of the Markov chain (interpreted as matrices) and $Z$ the normalization constant. From \cite{beal2003variational}, we have that variational updates should give
\begin{align}
    \tilde{A}_k &\leftarrow \overline{\log A_k} &
    \tilde{\pi}_k &\leftarrow \overline{\log \pi_k}
\end{align}
while the effective emission probabilities are
\begin{align}
    \eta_{tk} &\leftarrow \delta_{z_{tk}, 1}\cdot \sum_{m; t(m) = t} N_m \overline{\log \lambda_{zu(m)k}} - \sum_{m; t(m) = t} \overline{\theta_m} H_{0u(m)} F_{tku(m)} G_{tu(m)}
\end{align}
Given these update rules, we can then alternate between calculating $(\eta, \tilde{A}, \tilde{\pi})$, performing forward-backward to get $(\xi, \Xi, \log Z)$ and recalculating $(\eta, \tilde{A}, \tilde{\pi})$.

\subsection{Overdispersion, firing rate effects}
Both the case of $p(\theta)$ and $p(\lambda)$ are straightforward. If we ignore subscripts and write $p(y) = \text{Ga}(a, b)$, $q(y) = \text{Ga}(\alpha, \beta)$, then
\begin{equation}
    \mathbb{E}_q \left[\log \frac{p(y)}{q(y)} \right] = 
    (\overline{a} - 1) \overline{\log y} + \overline{b} \overline{y} + \mathcal{H}[q(y)]
\end{equation}
where again, $\overline{\log y}$, $\overline{y}$ and $\mathcal{H}[q(y)]$ are straightforward properties of the gamma distribution. For the expectations of the prior parameters, we have
\begin{align}
    \overline{a_{\theta u(m)}} &= \overline{s_{u}} & \overline{b_{\theta u(m)}} &= \overline{s_{u}} \\
    \overline{a_{0 u}} &= \overline{c_{0}} & \overline{b_{0 }} &= \overline{c_{0}}\overline{d_{0}} \\
    \overline{a_{z uk}} &= \overline{c_{zk}} & \overline{b_{z k}} &= \overline{c_{zu}}\overline{d_{zu}} \\
    \overline{a_{xur}} &= a_{xur} & \overline{b_{xur}} &= b_{xur}
\end{align}
since we assume no hierarchy on the firing rate effects for the covariates $x$. As we will see below, the expectations under the variational posterior of $s$, $c$, and $d$ are straightforward to calculate.

\subsection{Hyperparameters}
As shown in the main text, the hyperparameters $c$ and $d$, given gamma priors, have conjugate gamma posteriors, so that their contribution to the evidence lower bound, $\mathbb{E}_q \left[\log \frac{p(\gamma)}{q(\gamma)} \right]$ is a sum of terms of the form
\begin{align}
    (a_{c} - 1) \overline{\log c} + b_{c} \overline{c} + \mathcal{H}[q(c)] + 
    (a_{d} - 1) \overline{\log d} + b_{d} \overline{d} + \mathcal{H}[q(d)] 
\end{align}
In other words, these are straightforward gamma expectations, functions of the prior parameters $a$ and $b$ for each variable and the corresponding posterior parameters $\alpha$ and $\beta$. Similarly, the overdispersion terms are exactly the same with the substitutions $a \rightarrow s$, $d\rightarrow 1$.

\subsubsection*{References}
\begingroup
\renewcommand{\section}[2]{}
\bibliography{pearson_beck}{}
\bibliographystyle{ieeetr}
\endgroup
\end{document}